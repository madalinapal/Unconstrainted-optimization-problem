\documentclass{article}

\usepackage[utf8]{inputenc}
\usepackage{amsmath}
\usepackage{graphicx}
\usepackage{hyperref}
\usepackage{longtable}
\usepackage[a4paper, left=1cm, right=1cm, top=1cm, bottom=1cm]{geometry}

\title{Optimizarea re\c{t}elelor neuronale pentru analiza riscului de diabet}
\author{M\u{a}d\u{a}lina-Ioana Palade}
\date{\today}

\begin{document}

\maketitle

\section{Descrierea proiectului}
Acest proiect se concentreaz\u{a} pe dezvoltarea unui model de \^{i}nv\u{a}\c{t}are aprofundat\u{a} pentru clasificarea riscului de diabet, utiliz\^{a}nd o re\c{t}ea neuronal\u{a} cu un singur strat ascuns. Scopul principal este de a analiza indicatorii de s\u{a}n\u{a}tate \c{s}i stilul de via\c{t}\u{a} al pacien\c{t}ilor pentru a prezice dac\u{a} ace\c{s}tia sunt diabetici, pre-diabetici sau s\u{a}n\u{a}to\c{s}i, pe baza unui set de date ce include 35 de caracteristici medicale \c{s}i demografice.

Modelul va implementa optimizarea neconstr\^{a}ns\u{a}, aplic\^{a}nd cel pu\c{t}in dou\u{a} metode de optimizare studiate la curs, precum Metoda Gradientului \c{s}i Metoda Newton. Vor fi analizate performan\c{t}ele fiec\u{a}rei metode \^{i}n termeni de timp, convergen\c{t}\u{a} \c{s}i eficien\c{t}\u{a} \^{i}n cadrul antren\u{a}rii modelului.

Prin utilizarea func\c{t}iei de activare personalizat\u{a} 
\begin{equation}
g(z) = \cos(z) - z
\end{equation}
, proiectul va explora comportamentele re\c{t}elei neuronale \^{i}n raport cu diferitele tipuri de optimizare, \^{i}n scopul de a maximiza acurate\c{t}ea clasific\u{a}rii \c{s}i a \^{i}n\c{t}elege rela\c{t}ia dintre stilul de via\c{t}\u{a} \c{s}i riscul de diabet.

\section{Baza de date}

\begin{longtable}{|p{4cm}|p{2cm}|p{2cm}|p{6cm}|p{3cm}|}
\hline
\textbf{Numele Variabilei} & \textbf{Rol} & \textbf{Tip}  & \textbf{Descriere} & \textbf{Unități} \\
\hline
\endfirsthead
\hline
\textbf{Numele Variabilei} & \textbf{Rol} & \textbf{Tip} & \textbf{Descriere} & \textbf{Unități} \\
\hline
\endhead
\hline
\endfoot
\hline
ID & ID & Întreg  & ID-ul pacientului & nu \\
\hline
Diabetes\_binary & Țintă & Binare  & 0 = nu diabet, 1 = prediabet sau diabet & nu \\
\hline
HighBP & Caracteristică & Binare &  0 = nu tensiune arterială mare, 1 = tensiune arterială mare & nu \\
\hline
HighChol & Caracteristică & Binare &  0 = nu colesterol mare, 1 = colesterol mare & nu \\
\hline
CholCheck & Caracteristică & Binare &  0 = nu test colesterol în ultimii 5 ani, 1 = test colesterol în ultimii 5 ani & nu \\
\hline
BMI & Caracteristică & Întreg &  Indicele de masă corporală & nu \\
\hline
Smoker & Caracteristică & Binare &  0 = nu, 1 = da & nu \\
\hline
Stroke & Caracteristică & Binare &  0 = nu, 1 = da (accident vascular cerebral) & nu \\
\hline
HeartDiseaseorAttack & Caracteristică & Binare &  0 = nu, 1 = da (boli cardiace sau infarct) & nu \\
\hline
PhysActivity & Caracteristică & Binare &  0 = nu, 1 = da (activitate fizică în ultimele 30 de zile) & nu \\
\hline
Fruits & Caracteristică & Binare &  0 = nu, 1 = da (consum fructe 1 sau mai multe ori pe zi) & nu \\
\hline
Veggies & Caracteristică & Binare &  0 = nu, 1 = da (consum legume 1 sau mai multe ori pe zi) & nu \\
\hline
HvyAlcoholConsump & Caracteristică & Binare &  0 = nu, 1 = da (consum excesiv alcool) & nu \\
\hline
AnyHealthcare & Caracteristică & Binare &  0 = nu, 1 = da (acoperire de asigurare de sănătate) & nu \\
\hline
NoDocbcCost & Caracteristică & Binare &  0 = nu, 1 = da (nu s-a putut consulta un medic din cauza costurilor) & nu \\
\hline
GenHlth & Caracteristică & Întreg &  1 = excelent, 2 = foarte bun, 3 = bun, 4 = corect, 5 = slab & nu \\
\hline
MentHlth & Caracteristică & Întreg &  Număr de zile cu sănătate mentală proastă în ultimele 30 de zile & nu \\
\hline
PhysHlth & Caracteristică & Întreg &  Număr de zile cu sănătate fizică proastă în ultimele 30 de zile & nu \\
\hline
DiffWalk & Caracteristică & Binare &  0 = nu, 1 = da (dificultăți mari la mers sau urcat scări) & nu \\
\hline
Sex & Caracteristică & Binare &  0 = femeie, 1 = bărbat & nu \\
\hline
Age & Caracteristică & Întreg &  Categorie de vârstă (13 nivele) & nu \\
\hline
Education & Caracteristică & Întreg &  Nivel de educație (scale 1-6) & nu \\
\hline
Income & Caracteristică & Întreg &  Venit (scale 1-8) & nu \\
\hline
\end{longtable}

\section{Results}
Present the results or findings of your work. Include any relevant tables, graphs, or figures.

\section{Conclusion}
Summarize your work and discuss any conclusions or insights you gained.

\end{document}
